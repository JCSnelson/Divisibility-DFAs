\documentclass{article}
\usepackage{graphicx} % Required for inserting 
\usepackage{amsmath,amsfonts,amssymb}
\usepackage{float}
\usepackage{amsthm}
\newtheorem{theorem}{Theorem}[section]
\newtheorem{lemma}[theorem]{Lemma}
\title{Minimal DFA's for Divisibility Testing (LSB first)}
\author{Jez Snelson \& Joshua Obayomi}
\begin{document}
\maketitle
\section{Introduction}
\section{Non-Distinguishability Criteria}
% Declare function r_b(x) at some point
% Declare distinguishability equivalence relation for divisibility by p ~_{d,p}
\subsection{Initial ND Criteria}
\begin{lemma}
  % Define notation of x = val(Sx)
  For all $Sx,Sy\in\Sigma^*$ $Sx=_{d,p}Sy$ if and only if\\
  $\forall d\in\mathbb N$ $r_2(x)d+x\equiv0\iff r_2(x)d+y\equiv0$
\end{lemma}
\subsection{Revised ND Criteria}
\begin{lemma}
  For all $\alpha\in\mathbb Z/p\mathbb Z$ there exists $\alpha^{-1}\in \mathbb Z/p\mathbb Z$ if $a,p$ are coprime and $a\neq0$
\end{lemma}
\begin{proof}
If we pick $a$ and we have that $\alpha$ and $p$ are coprime we have by Bezout's identity we have that there exists integers $x$ and $y$ such that $\alpha x+py=1$ which implies that\\ 
\[\alpha x+py\equiv\alpha x+0\equiv\alpha x\equiv1\text{   (mod }p)\]\\
And so we take $\alpha^{-1}=x$ 
\end{proof}
\begin{lemma}
$(r_2(x))^{-1}x\equiv (r_2(y))^{-1}y \text{  (mod p)}$ if and only if\\
$\forall d\in\mathbb N$ $r_2(x)d+x\equiv0\iff r_2(x)d+y\equiv0$
\end{lemma}
\section{Equivalence Relation Classes}
As we have shown our distinguishability equivalence relation $_{d,p}$ is equivalent to $r_2(y)x\equiv r_2(x)y \text{  (mod }p)$ and we want to construct our distinguishing set from this which leads us to.\\
\begin{lemma}
  The amount of equivalence classes under $=_{d,p}$ is exactly $p$\\
  Also said as $|\{[Sx]_{d,p}|Sx\in\Sigma^*\}|=p$
\end{lemma}
\begin{proof}
  Firstly since there is only $p$ possible values for the numbers to be congruent to mod as they are integers we have that the amount of equivalence classes is $\leq p$.\\
  Now all we need to do is find $p$ possible equivalence classes so that 
\end{proof}

\section{Even Numbers}

\end{document}